\documentclass[12pt]{article}

\usepackage{sbc-template}

\usepackage{graphicx,url}

\usepackage[brazil]{babel}   
%\usepackage[latin1]{inputenc}  
\usepackage[utf8]{inputenc}  
% UTF-8 encoding is recommended by ShareLaTex

\sloppy

\title{Turbo Sansa}

\author{Daniel A. Amarante, Diego P. da Jornada}


\address{Faculdade de Informática -- Pontifícia Universidade Católica do Rio
    Grande do Sul\\  (PUCRS)\\
  \email{\{daniel.amarante, diego.jornada\}@acad.purs.br}
}

\begin{document} 

\maketitle

\begin{abstract}
  This paper was written as the documentation of the second project of the Computer Graphics I class of PUCRS. It describes a game about a lonely Teapot that has to collect cubes and spheres as fast as it can. The game was written in C++ and uses OpenGL for graphics.
\end{abstract}
     
\begin{resumo} 
  Este artigo foi escrito como a documentação do segundo projeto da disciplina de Computação Gráfica I da PUCRS. Ele descreve um jogo pode de chá que precisa coletar cubos e esferas o mais rápido o possível. O jogo foi escrito em C++ e utiliza do OpenGL para os gráficos.
\end{resumo}


\section{Informações gerais sobre o jogo}

O jogo consisde de um Teapot em um mundo 3d que tem como objetivo coletar o maximo de objetos antes do tempo acabar. Para coletar os objetos basta andar sobre eles. Cada objeto diferente dá ao player uma pontuação diferente, cubos dando um ponto cada e esferas dando cinco pontos cada, o objetivo do jogo é juntar o máximo de pontos antes do tempo acabar.

\section{Representação}

Os objetos do jogo são representados por um vetor de posições para saber onde desenhá-los, um vetor de tipo, para decidir se são esferas ou cubos e um vetor booleano que representa se já foi coletado ou não. Um objeto já coletado some do mapa, e passar por aquele ponto não garante ao jogador mais pontos.

\section{Formas}

Os objetos do jogo assumem diversas formas, o jogador é um Teapot e os objetos coletáveis são esferas e cubos. Todos os objetos são criados automáticamente pelo opengl com suas funções de \emph{GlutSolidTeapot}, \emph{GlutSolidCube} e \emph{GlutSolidSphere}. Os objetos são posicionados sobre um chão desenhado de tamanho 1000x1000. São 1000 objetos ao todo sem contar o player e o chão.

\section{Movimentação}

A movimentação é feita utilizando-se mouse e teclado, o jogador se move e leva a camera consigo, tendo o jogo a visão de terceira pessoa. O teclado translada o jogador pelo mapa para frente, para trás e para os lados e o mouse rotaciona a câmera. Foi implementada uma maneira de fazer dois botões funcionarem ao mesmo tempo por meio de um vetor de teclas apertadas, podendo o jogador andar para frente e para o lado ao mesmo tempo (na diagonal).

\section{Colisões}

As colisões acontecem quando o jogador se aproxima o suficiente de um outro objeto, estando o centro do jogador a 1 ponto de distância do objeto esse objeto é "coletado", sumindo do mapa e garantindo ao player pontos correspondentes ao seu valor.

\end{document}
